\documentclass[12pt]{article}
\usepackage{fbb}

\makeatletter
\setlength{\@fptop}{0pt}
\makeatother
\usepackage{graphicx}
\usepackage{fancyhdr}
\setlength{\topmargin}{0.5in}
\setlength{\headheight}{15pt}
\setlength{\headsep}{0.25in}
\setlength{\textheight}{9.5in}
\setlength{\textwidth}{7in}
\setlength{\oddsidemargin}{0in}
\setlength{\evensidemargin}{0in}
\setlength{\parindent}{0.2in}
\setlength{\parskip}{0.2in}
\pagestyle{fancy}
\renewcommand{\headrulewidth}{0.5pt}
\renewcommand{\footrulewidth}{0.5pt}
\usepackage[margin=2cm]{geometry}
\renewcommand{\headrulewidth}{0pt}

\fancyfoot[RE,LO]{Math circles with Vijay}
\fancyfoot[RO,LE]{October 25, 2014}

\begin{document}
\begin{figure}[t!]
  \centering
  \includegraphics[scale=0.5]{./cubes.jpg}
 \end{figure}
 \vspace{-40pt}
{\Large \centering \hspace{0.35\textwidth} Cubes, Cubes, Cubes!}
\date{\vspace{-10ex}}

\begin{enumerate}
\item Your math circles instructor brought a box containing a large number of 1$\times$1$\times$ 1 wooden cubes and introduced a game with some simple rules:
\begin{itemize}
\item If you persuade another circler to accept a cube from you, you can pocket a cube. 
\item You can do this with a person only once - if somebody already gave you a cube, you cannot give them one. 
\item If you end up with an odd number of cubes, you can take your cubes home with you. Otherwise, you give it back to your instructor.
\end{itemize}

What is the maximum number of circlers who can go home with their cubes? How many cubes did the instructor have to part with?

\item You ended up with 27 cubes and decided to make a 3$\times$ 3$ \times$ 3 cube and keep it in front of you. Your math circles instructor sees this and asks you - A \emph{tiny} ant starts at the cube on the bottom left corner and makes its way towards the top corner. If the ant always takes a unit step along one of the three mutually perpendicular directions defined by the cube, how many different paths can the ant reach the top corner?


\item After a fun day at math circles, you went home and kept the 3$\times$ 3$ \times$ 3 cube on your table. A couple of days later, you woke up to find a termite just about starting to bore in from the centre small cube on the top face of your big cube. You immediately shook the termite off your precious cube and blame your instructor and his ant problem for its appearance.

Later on, you mention this incident to a fellow math circler; who, after some thinking, asks you the following question. If the termite started where it was, and could only bore parallel to the faces of the cube, could it end up at the center cube having visited each small cube only once?

What would your answer be if you had ended up with a different number of cubes from the game that could be put together to make a bigger cube?
\end{enumerate}
%\author{Math circles with Vijay}
%\date{Oct 25,2014}
\end{document}